\chapter{总结与展望}
\section{全文总结}
通观全文,与其说实在研究如何提高人脸属性的识别准确率,倒不如说是在各种偏离基础算法使用场景的情况下,解决一个又一个出现的问题,包括
为了能够提高训练速度,在训练中的不同框架,尝试探究多机多卡。
为了在实际测试中,具有较高的反馈和实用价值,在基础的神经网络操作中,对于基本算法的加速和改进。
为了适应针对数据集训练和评测的这种模式,设计针对于多种属性标签,多种数据集的网络架构。
为了对于不同的场景数据分布存在偏差的问题,针对于背景的变化,使用对抗生成网络构建超分辨率系统对于人脸图片进行了人脸的跨域重构,从而提高目标域的识别效果。
在这个过程中,对于图像中模式识别的基本算法有所掌握,同样也印证着发现问题,分析问题和解决问题的思路。

从学术研究的贡献来看,论文中主要是对于现有算法更加深入和具体化的改进。从整个研究任务的完成上,主要的思考点和实现的标准有两方面:

一方面从底层实现上探究在目前计算机水准上,热衷于探究对于算法的实现能够具有怎样的加速方法,让算法的实现和使用变得快速化。另一方面在常规的网络构建上思考如何能够构建更加具有端到端的特性,让机器学习的相关问题分析和解决流程变得更加简洁化。


\section{未来展望}
本文所介绍的人脸属性识别属于图像识别种基于监督学习的分支,同时也是非常具有代表性的任务之一,类似的人物包括物体识别中的物体性质识别,如经典的鱼种类识别等。
所以人脸属性识别的进展需要依托于整个图像识别的基础技术进展和图像数据库的建设。而图像识别领域基础技术的进展其实更加依托于更加基础计算机科学的进步与发展,细节小到晶体管的制造工艺,计算机内存和缓存的读取速度,处理器的主频提升,布局大到整个体系结构的变革,冯诺依曼体系的变革,量子计算机的进化等,都会对于模式识别算法有着较为深远的影响。

除了对于底层科学的依赖,现实生活中的应用落地也同样具有重大意义,比如慢慢成熟的人脸识别,自动驾驶等新兴技术行业,无一不是有基础的模式识别技术发展而来,但是却无一不在现实的产业结构中引发巨大热潮,让实验室的算法走出实验室出现在人们的现实生活种,可以极大激励人们对于人工智能的探索的热情和改变人类生活状态的前行动力。并且在实际生活中慢慢探索图像识别的规律,加速人工智能领域的快速发展。



 
