%%
%% This is file `example/metadata.tex',
%% generated with the docstrip utility.
%%
%% The original source files were:
%%
%% install/buptgraduatethesis.dtx  (with options: `metadata')
%% 
%% This file is a part of the example of BUPTGraduateThesis.
%% 

%% 涉密论文保密年限
\classdur{三年}

%% 学号
\studentid{2015140024}

%% 论文题目
\ctitle{基于对抗神经网络的人脸图片属性识别与生成}
\etitle{Generative Adversarial Network based Face attribute recognition and regeneration}

%% 申请学位
\cdegree{工程硕士}

%% 院系名称
\cdepartment{信息与通信工程学院}

%% 专业名称
\cmajor{电子与通信工程}

%% 你的姓名
\cauthor{于志鹏}

%% 博士后研究工作报告-分类号
\classnumber{O441.3}

%% 博士后研究工作报告-UDC
\udc{621.396.9}

%% 博士后研究工作报告-学校编号
\schoolserial{147227}

%% 博士后研究工作起始时间
\startdate{2014年10月29日}

%% 博士后研究工作期满时间
\finishdate{2016年4月2日}

%% 你导师的姓名
\csupervisor{董远}

%% 日期自动生成,也可以取消注释下面一行,自行指定日期
\cdate{\CJKdigits{2017}年\CJKnumber{11}月\CJKnumber{30}日}

%% 中文摘要
\cabstract{%
在模式识别与多媒体搜索领域,深度学习卷积神经网络是近年的新兴技术,凭借着简洁、有效、易训练等优势迅速在图像处理领域得到了广泛的应用。尤其是在人脸相关的领域,卷积神经网络的出现极大提升了人脸识别和人脸属性识别的准确率,俨然已经成为目前人脸领域的主流技术和最具前景的技术方向。	作为神经网络另类演变,对抗生成网络初期是为了探究神经神经网络的内部构造原理。随着相关技术的不断进化,借助其可以生成逼真图像的特性,在图像重建领域也体现出了很强的实用价值。与此同时,随着监督式学习的性能瓶颈到来,迁移学习的理念作为非监督学习与监督学习之间的过渡,提出可以通过现有场景的数据和方法来探索未知场景下的识别任务,体现出具有较高的研究意义。
	
本文在工程实践和理论研究方面有所兼顾,首先介绍了卷积神经网络的基础结构,传统的理论训练和测试方法;随后介绍了工程上如何使用分布式多卡训练加快神经网络的训练;实际的生产中对于网络前馈所用到的优化技巧,诸如卷积优化,多个计算步骤合并等。最后借助于深度学习框架和指令集技术可以大幅提高训练速度以及10倍的前馈速度提升。
	
研究方面,本文主要研究人脸属性基础识别和人脸属性的迁移学习两个方面的问题。基础的人脸属性识别存在着对于多个数据集任务难以共同利用和网络输出可信程度的把控两个问题,通过调整网络的结构、改进图片预处理的方式、设计自评模块等方法对于相关问题进行了针对性地解决,也提升了人脸图片的属性识别准确性。在morph年龄数据集上绝对误差只有3.5岁,在chalearn fotw性别和微笑数据集准确率上也超过了90\%,而在实验室自己标注的的5类年龄数据集top1准确率达到了93.6\%。
	
另一方面结合对抗生成网络探究迁移学习在人脸属性上的应用,首先通过对于人脸图片进行生成,不断优化合成数据的真实度和广泛性,获得了可以生成真实人脸的神经网络。而为了让人脸属性模型应用于不同的的使用场景,借助于人脸超分辨率的技术结合迁移学习的思想构建了可以对于celeA和lfwA两个数据集都具有良好表现的40类人脸属性模型训练方法。相比于原有模型提高了10个百分点。
}

%% 中文关键词,关键词之间用 \kwsep 分割
\ckeywords{\kwsep 对抗生成网络 \kwsep 人脸属性 \kwsep 迁移学习 \kwsep 多机多卡 \kwsep 前馈优化 }

%% 英文摘要
\eabstract{%
In the field of pattern recognition and multimedia search, deep learning convolutional neural network is an emerging technology in recent years, but it has been widely applied in the field of image processing due to its advantages of conciseness, efficiency and easy training. Especially in the face-related fields, the emergence of convolutional neural networks has greatly improved the accuracy of face recognition and face recognition, which has become the mainstream technology and the most promising technical direction in the current face area. As an alternative to the evolution of neural networks, the early days of confrontation generation networks were to explore the internal construction of neural networks. With the continuous evolution of related technologies, with its characteristics that can generate realistic images, it also shows strong practical value in the field of image reconstruction. At the same time, with the arrival of performance bottleneck in supervised learning, the concept of Migration Learning, as a transition between unsupervised learning and supervised learning, proposes that data and methods of existing scenes can be used to explore recognition tasks in unknown scenarios. It has a high research significance.

In this paper, both engineering practice and theoretical research take into account, first introduced the convolution neural network infrastructure, the traditional theory of training and testing methods; then introduced how to use the project to accelerate the training of distributed multi-card neural network training; actual The optimization techniques used for network feedforward, such as convolution optimization, multiple calculation steps and so on. Finally, with the help of deep learning framework and instruction set technology, it can achieve 4 times of training speed improvement and 10 times of feedforward speed increase compared with the common method.

In the aspect of research, this dissertation mainly studies two aspects of the basic recognition of face attributes and the migration of face attributes. The basic recognition of face attributes is based on the difficulty of using multiple data sets and the credibility of network output By adjusting the structure of the network, improving the way of image preprocessing and designing self-evaluation modules, the two problems are solved in a targeted way and the accuracy of attribute recognition of face pictures is also improved. The absolute error in the morph age dataset was only 3.5 years old, more than 90% in the chalearn fotw gender and smile dataset accuracy, and the top 5 accuracy of the 5 age-related datasets annotated by the laboratory reached 93.6% %

On the other hand, we combine the countermeasure generation network to explore the application of relocation learning to face attributes. Firstly, we generate the neural network that generates the real face by generating the face images and optimizing the authenticity and universality of the synthesized data. In order to apply the facial attribute model to different usage scenarios, the 40 types of face attributes that can perform well on both celeA and lfwA datasets are constructed with the help of the theory of face-super-resolution combined with migration learning Model training methods. Compared to the original model increased by 10 percentage points.

}

%% 英文关键词,也用 \kwsep 分割
\ekeywords{%
   \kwsep GAN \kwsep face attribue \kwsep transfer leearning \kwsep Multi-machine multi-card \kwsep Feedforward optimization}
