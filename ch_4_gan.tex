\chapter{对抗生成网络在人脸属性中的应用}
\section{对抗生成网络相关技术的介绍}
早在2014年,人们对于神经网络技术的研究非常狂热的同时,也有一部分理智的科学家认为神经网络的输出判断具有非常高的风险,输出具有非常高的不稳定性,所谓数据集上的准确率超越人类不过是一场谎言,为了戳穿这一谎言,他们在神经网络判断正确的图片上简单加了一些噪声,对于人类来说根本没有察觉图像的变化,但是在神经网络却完全将其判断成另外一种物体。同时科学家们宣称这样极具欺骗性的图片并非偶然得到,而是可以量产的,比如通过GAN网络。

借助于博弈论中的零和博弈思想(在零和博弈中,游戏玩家之间的利益总和是固定的,即一方获得收益,另一方就要承担损失。)
Goodfellow极具想象力的提出了可以通过搭建两个对抗的网络,各自的目的就是降低对方的准确率,或者说提升对方loss。通过这样非常具有竞争性的训练过程,最够提升两个网络的性能。
具体来讲:在对抗生成网络中,玩家的角色会分别有生成模型(generative model)和判别式模型(discriminative model)充当.生成模型G捕捉样本数据的分布,判别模型D是一个二分类器,估计一个样本来自于训练数据(而非生成数据)的概率。
G和D可以是线性代数的算法操作组合,也可以是神经网络的网络模型,都可以理解成或者定义成非线性函数。
通过不断调整G和D,直到D不能把事件区分出来为止。
在调整过程中,需要:
优化G,使得它尽可能的让D混淆
优化D,使得它尽可能的能区分出假冒的东西
当D无法区分出事件的来源的时候,可以认为,G和M是一样的。从而,就获得了能够以假乱真的数据。

而在不断的发展中GAN网络有了更多的应用和算法分支
\subsection{非监督图片的生成}

\subsection{图片超像素}


\section{对抗神经网络在人脸属性中的应用}
从上面对抗生成网络的演变和发展来看,对抗生成网络很明显并不能像正常的CNN网络一样对于具体的模式识别任务,但是作为探究CNN生成原理的一部分,对抗生成网络主要是希望能够了解CNN能够从图像中学习到什么样的信息,怎样学习的,并且能否以较为直观的形式也就是生成图像来表示出来,(尽管学习到的东西很多时候并不能够以图像的形式进行展现)。
在本论文的实验后期,我们想到了使用对抗生成网络来探究一下CNN对于不同分布的数据集之间的学习能力。
\subsection{人脸属性的监督式学习困境}
首先引入一个经典的模式识别场景,泛化能力的问题:
在Hu的工作种,他发现一个在很多实验中都会出现的问题,使用MTL的人脸属性框架进行人脸属性识别的过程中,具同样40个属性标签的两个数据集lfwA和celeA,两个在各自数据集上训练之后的模型,在各自数据集上的准确率都很高,但是在对方的测试集效果都比较糟糕。
(加入hu的lfw和celeA的实验对比表)
如何进行改善呢?我们针对于这种情况设计了这样的思路:
问题引出:对于相同的网络模型,使用相同的训练方法,在不同数据集中的训练之后,对自身数据集的测试集准确率要远远高于其他数据集的测试集。
问题分析:首先这不是一个过拟合问题,因为对于数据集中训练集和测试集的准确率较高,所以网络的训练没有问题。但是对于不同数据集的测试集准确率很低,所以推测问题的出现是因为数据的分布不同

尝试解决办法:首先我们先假定网络模型容量可以容纳两个数据的分布(数据的分布可能不满足线性加法,但是应该满足集合性合并不减的特性,所以假定两种数据的分布集合会比原来更大,所以对于网络容量的要求会更大),既然数据的分布不同,就应当减少数据分布对于模型训练带来的影响。

第一种方法就是将两个数据集合并训练,如果标签相同,那么可以简单的将两个数据集合并成一个数据集训练,也可以首先在一个数据集上份训练,再经过另一个数据集finetune,又或者采用上一章所提到的主干网路参数共享,不同数据集分别使用一个网络支线进行训练。都可以直观地学习到两个数据集之间的数据分布。往往就可以取得较好的效果,有效的提高在不同数据集上准确率的表现。
缺点:最致命的缺点就在于不同数据集的准确率提高,但是难以保证在自身的数据集上数据的准确性。即使采用较小的学习率谨慎的进行finetune,对于不同任务的训练过程也即将面临着大量的手动干预,还是处于一个监督学习的框架之中。对此我们决定使用类似于迁移学习的方式来完成这个任务,并且结合gan网络来完成我们的任务。具体思路是这样的:
从上面对于gan网络的介绍中可以发现,
\subsection{使用GAN网络生成训练数据以扩充监督式学习方法}
在之前对于GAN网络的介绍中,可以发现GAN网络最初是来证明神经网络算法对于数据分布具有一定的局限性。而慢慢发展,人们并不在乎神经网络是否对于数据分布有一定的的局限性,而狂热的希望能够通过GAN网络获得以假乱真的机器生成图片。似乎人们觉得如果机器能创造他,那机器肯定可以了解他,那么识别他也是轻而易举。于是乎这种炫酷,但是有一定投机取巧性质的思路不仅开始影响最初使用GAN网络探究神经网络有效性的本意,也影响着各种识别任务的传统数据+模型的预测方式。
针对于这种非常具有诱惑力的尝试方式,即通过使用噪声实现对于特定图像的无标注成本转换,我们制定了如下的网络结构

首先实现了从100维的噪声生成数字图像,

然后实现了32*32的物体图像,

甚至实现了具有一定属性人脸图像。

但尽管证明GAN网络确实能够自动的生成具有一定真实图片特征的图片但是生成图片的效果还是没有较好的方式来进行控制,远远不能达到以假乱真的程度。
与此同时使用GAN网路生成图片,然后加入训练数据中的思想,还是局限于监督学习的框架,需要具有标注的网络图像。

\subsection{结合GAN超像素实现迁移学习}
在发现GAN网络其实并不能直接从噪声生成具有一定训练意义的图片之后,我们并没有气馁。在参考了很多具有使用意义的GAN网络工作之后,决定从超像素的方向重新研究。











